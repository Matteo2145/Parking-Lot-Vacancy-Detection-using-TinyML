\documentclass[sigconf,10pt,nonacm]{acmart}

\hfuzz=100pt
\hbadness=99999
\vbadness=99999

\renewcommand\footnotetextcopyrightpermission[1]{}

\pdfstringdefDisableCommands{%
  \def\\{}%
}

\usepackage{hyperref}


%% \BibTeX command to typeset BibTeX logo in the docs
\AtBeginDocument{%
  \providecommand\BibTeX{{%
    Bib\TeX}}}

%% Rights management information.  This information is sent to you
%% when you complete the rights form.  These commands have SAMPLE
%% values in them; it is your responsibility as an author to replace
%% the commands and values with those provided to you when you
%% complete the rights form.
\setcopyright{acmlicensed}
\copyrightyear{2024}
\acmYear{2024}
\acmDOI{XXXXXXX.XXXXXXX}


\begin{document}

\title[Project/Paper Title]{Project/Paper Title}

\author{Name Surname}
\email{xyz@unitn.it}
\orcid{123}


\affiliation{%
  %\institution{Department of Information Engineering and Computer Science}
  \institution{University of Trento}
  \streetaddress{Via Sommarive 9}
  \city{Povo}
  \state{Trento}
  \country{Italy}
  \postcode{38123}
}

\renewcommand{\shortauthors}{XXX et al.}


\begin{abstract}
Write a short summary of the work. 
\end{abstract}

\maketitle

I \emph{recommend}---but do not require--- that you use
the following organization for your abstract. 
Sections~\ref{sec:motivation} through \ref{sec:key-contributions}
should be a summary of your full paper. Section~\ref{sec:motivation}
motivates the paper; Section~\ref{sec:limitations} describes
limitations of the state of the art, if applicable;
Section~\ref{sec:key-insights} presents the key new insight or
insights of the paper; 
Section~\ref{sec:main-artifacts} presents the main artifacts described
in the paper;  Section~\ref{sec:key-contributions} summarizes the key
results and technical contributions of your paper. 

\begin{itemize}
    \item Your paper should be {\bf minimum 4} pages.

    \item You will submit your paper via {\bf \href{https://unitn-lpes24.hotcrp.com/}{https://unitn-lpes24.hotcrp.com/}}
\end{itemize}


\section{Motivation}
\label{sec:motivation}


\begin{itemize}
\item What is the problem your work attacks? Be specific.
\item Why is it an important problem?
\end{itemize}

\noindent
Articulate the importance of this problem, using as little jargon as possible. \emph{Be specific}
about the problem you are addressing; it should be the one that your
paper directly addresses.

\section{Limitations of the State of the Art}
\label{sec:limitations}

\begin{itemize}
\item What is the state of the art in this topic today (if any)?
\item What are its limits?
\end{itemize}

\section{Key Insights}
\label{sec:key-insights}

\begin{itemize}
\item What are the one or two key new insights in this paper?
\item How does it advance the state of the art?
\item What makes it more effective than past approaches?
\end{itemize}

\section{Main Artifacts}
\label{sec:main-artifacts}

\begin{itemize}
\item What are the key artifacts presented in your paper: a
  methodology, a hardware design, a software algorithm, an
  optimization or control technique, etc.?
  \item How were your artifacts implemented and evaluated? 
\end{itemize}

\section{Key Results and Contributions}
\label{sec:key-contributions}

\begin{itemize}
  \item What are the most important \emph{one or two} empirical or theoretical
    results of this approach?
  \item What are the contributions that this paper makes to the state of the
    art? List them in an \texttt{itemize} section. Each contribution should be no more than a few sentences long.
  \item Clearly describe its advantages over past work, including how it overcomes their limitations.
\end{itemize}

This is reference~\cite{lamport94} example.

\bibliographystyle{plain}
\bibliography{references}


\end{document}